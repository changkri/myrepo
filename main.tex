%\documentclass[openany]{book}
\documentclass[12pt]{article}
\usepackage[utf8]{inputenc}
\usepackage{amssymb} %for fancy L
\usepackage{calrsfs} %for fancy L
\usepackage{cancel}
\usepackage{tabularx}
\usepackage[hyphens]{url}
\usepackage{booktabs}
\usepackage{graphicx}
\usepackage[titletoc,title]{appendix}
\usepackage{subfig}
\DeclareMathAlphabet{\pazocal}{OMS}{zplm}{m}{n} %for fancy L
\usepackage{epsfig, float,array,tabu,longtable,}
\usepackage{hyperref,wrapfig}
\usepackage{enumerate}
\usepackage{graphicx,psfrag}
\usepackage{cite}
\usepackage{sectsty}
\usepackage{epstopdf}
\usepackage{amsmath,esint, setspace, fancyhdr, amsfonts, bookmark, blindtext}
\usepackage[normalem]{ulem}
\usepackage{tikz}
\usepackage{rotating}
\usepackage[americanvoltages,fulldiodes,siunitx]{circuitikz}
\usepackage{stackengine}
\usetikzlibrary{matrix}
\usepackage{multirow}
\usepackage{multicol}
\usetikzlibrary{shapes,backgrounds,patterns}
\usetikzlibrary{mindmap,trees,decorations.markings}
\usetikzlibrary{quotes,angles}
\usepackage{verbatim}
\renewcommand{\baselinestretch}{1}
\setlength{\textheight}{8in}
\setlength{\textwidth}{6.5in}
\setlength{\headheight}{0in}
\setlength{\headsep}{0.25in}
\usepackage{graphicx}
\setlength{\topmargin}{0in}
\setlength{\oddsidemargin}{0in}
\setlength{\evensidemargin}{0in}
\setlength{\parindent}{.3in}
\usepackage{listings}
\usepackage{color} %red, green, blue, yellow, cyan, magenta, black, white
\definecolor{mygreen}{RGB}{28,172,0} % color values Red, Green, Blue
\definecolor{mylilas}{RGB}{170,55,241}
\doublespacing
\begin{document}


\begin{titlepage}

\newcommand{\HRule}{\rule{\linewidth}{0.5mm}} % Defines a new command for the horizontal lines, change thickness here

\center % Center everything on the page
 
%---------------------------------------------------------
%	HEADING SECTIONS
%---------------------------------------------------------

\textsc{\LARGE One versus two hour frequency of intentional rounding}\\[1.5cm] % Major Heading
\textsc{\Large A single-centre pragmatic multiple-crossover trial}\\[0.5cm] % Major heading s
\textsc{\large }\\[0.5cm] % Minor heading

%---------------------------------------------------------
%	TITLE SECTION
%---------------------------------------------------------

\HRule \\[0.6cm]
{ \huge \bfseries Research Protocol}\\[0.4cm] % Title of your document
\HRule \\[1.0cm]
 
%---------------------------------------------------------
%	AUTHOR SECTION
%---------------------------------------------------------

%\begin{minipage}{0.4\textwidth}
\begin{center} \large
% \emph{Authors:}  
\medskip
{\textsc{\textbf{Aaron Conway} }}  \quad  {\textsc{\textbf{Linda Flockhart}}}  \quad {\textsc{\textbf{Zelia Souter}}}
\quad {\textsc{\textbf{Dorina Baston}}} \quad {\textsc{\textbf{Joy Richards??}}} \quad {\textsc{\textbf{Barry Rubin??}}}% Your name, bold and small caps
\end{center}
%\end{minipage}

\begin{center} \Large
{\textsc{\textbf{Version 1} }} 
\end{center}

~
%\begin{minipage}{0.4\textwidth}
%\begin{flushright} \large
%\emph{Supervisor:} \\
%Dr. James \textsc{Smith} % Supervisor's Name
%\end{flushright}
%\end{minipage}\\[4cm]

% If you don't want a supervisor, uncomment the two lines below and remove the section above
%\Large \emph{Author:}\\
%John \textsc{Smith}\\[3cm] % Your name

%---------------------------------------------------------
%	DATE SECTION
%---------------------------------------------------------
\begin{center}
{\large \today}
\end{center}
 % Date, change the \today to a set date if you want to be precise

%---------------------------------------------------------
%	LOGO SECTION
%---------------------------------------------------------
%\vfill

\newcommand*{\plogo}{\includegraphics[width=0.5\textwidth]{image001.jpg}}

\plogo\\[1cm] % Include a department/university logo - this will require the graphicx package
 
%---------------------------------------------------------

\vfill % Fill the rest of the page with whitespace
\end{titlepage}

\newpage
\tableofcontents
\newpage

% EXECUTIVE SUMMARY %%%%%%%%%%%%%%%%%%%%%%%%%%%%%%%%%%
\section{Executive Summary}
\subsubsection*{Primary Objective}
To test the hypothesis that one hour intentional rounding reduces patient falls in comparison to two hour intentional rounding. 
\subsubsection*{Study Design}
A single-centre pragmatic multiple-crossover trial will be conducted. The frequency that nurses within participating departments at the Peter Munk Cardiac Centre are to perform intentional rounding will be randomly assigned as either one hour or two hours on a month to month basis (could be weekly instead). All other elements of the rounding process will remain the same.
\subsubsection*{Outcomes}
The primary outcome is the count of falls (reported in the incident reporting system) per patient hour. Secondary outcomes are:
\begin{itemize}
    \item Falls with injuries
    \item Pain ratings (from EPR)?
    \item Completeness of documentation of intentional rounding (random audit on subset of days)
    \item Missed vital sign measurements (EPR)
    \item Patient satisfaction (previous -low quality- research found that rounding increased satisfaction across all subscales of HCAHPS)
    \item Call bell requests to nursing staff (if feasible - random audit)
    \item Nutrition?
    \item Clinical deterioration (measured as the number of MET calls)?
\end{itemize}

\subsubsection*{Study Population}
All patients admitted to the Cardiology and Cardiovascular Surgery inpatient wards at the Peter Munk Cardiac Centre. 

\subsubsection*{Sample Size}
The study duration has been set to 12 months and the sample size will be the number of patients admitted to the study wards during that time. This study duration was chosen based on feasibility considerations and to ensure numerous crossovers, enrollment throughout the calendar year, allow adequate time for the study procedures into clinical care (i.e. the change in frequency of rounding) and to accumulate a large sample size adequate to detect important difference in outcomes between groups. Based on historical data from the Peter Munk Cardiac Centre, we anticipate a sample size of approximately \colorbox{yellow}{6000} patients distributed between the 1 and 2 hour intentional rounding groups.


\section{Background}




\section{Objective}
The objective of this study is to test the hypothesis that one hour intentional rounding reduces patient falls in comparison to two hour intentional rounding. 

\section{Methods}

This pragmatic, single-centre, multiple-crossover trial will be conducted for a specific period with no fixed sample size. A design with numerous short periods and frequent crossovers was selected to minimize the risk of changes over time in the patient population and usual care confounding trial results. The approach of using multiple crossovers in a pragmatic trial to compare the effects of different approaches used already applied in clinical practice is established and has been previously used for high profile studies recently published in the NEJM.\cite{self2018} 


We decided that it is suitable to compare the effects of different frequencies of intentional rounding using a crossover design because there is no risk of 'carryover' effects. Even if a patient is transferred to a different ward during their hospital admission, which has been assigned to a different intentional rounding frequency, there is no reason to suspect that the frequency of rounding performed for that patient earlier in the day will exert any difference in that patient's risk of falling later in the day when they are in a ward with the alternative intentional rounding frequency.  

Furthermore, compared with the previously used observational studies used that have been used to evaluate the effectiveness of intentional rounding on improving clinical, process/operational and patient-oriented outcomes, our trial design incorporates several features that reduce risk of bias. 

\subsection{Participants}
\subsubsection{Inclusion criteria}
\begin{itemize}
    \item 
\end{itemize}

\subsubsection{Exclusion criteria}
\begin{itemize}
    \item 
\end{itemize}

\subsection{Intervention and control conditions}
All patients....

How will the intentional frequency rounding be communicated to nurses to ensure they are adhering to the assigned conditions for the trial?

\subsubsection{One hour intentional rounding}

\subsubsection{Two hour intentional rounding}

\subsection{Outcomes}

\subsubsection{Primary}

\subsubsection{Seconday}

\subsection{Procedures}

\subsubsection{Enrolment}

\subsubsection{Randomization sequence generation}

\subsubsection{Allocation concealment}

\subsubsection{Data collection}

\subsection{Instruments}

\subsubsection{HCAHPS}
\subsubsection{Incident reporting system for falls}
\subsubsection{Electronic patient record}
\subsubsection{Call bell usage audit}

\subsection{Sample size considerations}
The study duration has been set to 12 months and the sample size will be the number of patients admitted to the study wards during that time. This study duration was chosen based on feasibility considerations and to ensure numerous crossovers, enrollment throughout the calendar year, allow adequate time for the study procedures into clinical care (i.e. the change in frequency of rounding) and to accumulate a large sample size adequate to detect important difference in outcomes between groups. Based on historical data from the Peter Munk Cardiac Centre, we anticipate a sample size of approximately \colorbox{yellow}{6000} patients distributed between the 1 and 2 hour intentional rounding groups.


\subsection{Statistical analysis plan}
 The primary analysis will be an intention-to-treat analysis of eligible patients assigned to one versus two hour intentional rounding based on the primary outcome of falls per days in hospital. 
\section{Funding}


% BIBLIOGRAPHY %%%%%%%%%%%%%%%%%%%%%%%%%%%%%%%%%%%%%%%
\newpage
\section{References}

\bibliographystyle{unsrt}
\bibliography{bibliography}
\fancyhead[R,OL]{bibliography}

% APPENDIX A %%%%%%%%%%%%%%%%%%%%%%%%%%%%%
\newpage
\section{Appendix A: Survey Instruments}



% APPENDIX H: BUDGET %%%%%%%%%%%%%%%%%%%%%%%%%%%%%%%%%%

\section{Appendix B: Complete Budget}

\end{document}
